% Options for packages loaded elsewhere
\PassOptionsToPackage{unicode}{hyperref}
\PassOptionsToPackage{hyphens}{url}
%
\documentclass[
]{article}
\usepackage{amsmath,amssymb}
\usepackage{lmodern}
\usepackage{iftex}
\ifPDFTeX
  \usepackage[T1]{fontenc}
  \usepackage[utf8]{inputenc}
  \usepackage{textcomp} % provide euro and other symbols
\else % if luatex or xetex
  \usepackage{unicode-math}
  \defaultfontfeatures{Scale=MatchLowercase}
  \defaultfontfeatures[\rmfamily]{Ligatures=TeX,Scale=1}
\fi
% Use upquote if available, for straight quotes in verbatim environments
\IfFileExists{upquote.sty}{\usepackage{upquote}}{}
\IfFileExists{microtype.sty}{% use microtype if available
  \usepackage[]{microtype}
  \UseMicrotypeSet[protrusion]{basicmath} % disable protrusion for tt fonts
}{}
\makeatletter
\@ifundefined{KOMAClassName}{% if non-KOMA class
  \IfFileExists{parskip.sty}{%
    \usepackage{parskip}
  }{% else
    \setlength{\parindent}{0pt}
    \setlength{\parskip}{6pt plus 2pt minus 1pt}}
}{% if KOMA class
  \KOMAoptions{parskip=half}}
\makeatother
\usepackage{xcolor}
\IfFileExists{xurl.sty}{\usepackage{xurl}}{} % add URL line breaks if available
\IfFileExists{bookmark.sty}{\usepackage{bookmark}}{\usepackage{hyperref}}
\hypersetup{
  hidelinks,
  pdfcreator={LaTeX via pandoc}}
\urlstyle{same} % disable monospaced font for URLs
\setlength{\emergencystretch}{3em} % prevent overfull lines
\providecommand{\tightlist}{%
  \setlength{\itemsep}{0pt}\setlength{\parskip}{0pt}}
\setcounter{secnumdepth}{-\maxdimen} % remove section numbering
\ifLuaTeX
  \usepackage{selnolig}  % disable illegal ligatures
\fi

\author{}
\date{}

\begin{document}

\hypertarget{just-to-be-sure}{%
\subsection{Just to be sure}\label{just-to-be-sure}}

for the teacher :-)

\hypertarget{pair-programming}{%
\subsubsection{Pair Programming}\label{pair-programming}}

\begin{itemize}
\item
  Pair of programmers develop higher-quality software
\item
  A little longer development time, but a lot less time spent finding
  bugs afterwards.
\item
  Cosier and more fun than working alone.
\item
  Improved knowledge sharing between programmers, less time spent on
  coordination
\item
  Increased learning (and that's why you're here, isn't it?), you take
  turns teaching each other
\item
  Otherwise, not pronounced knowledge and habits are exchanged.
\end{itemize}

\hypertarget{pair-programming-in-practice}{%
\subsubsection{Pair Programming in
Practice}\label{pair-programming-in-practice}}

\hypertarget{two-programmers-one-computer}{%
\paragraph{Two programmers, one
computer}\label{two-programmers-one-computer}}

\begin{itemize}
\item
  Driver - has hands on the wheel (keyboard) and eyes on the road
  (screen)
\item
  Navigator - focuses on the destination and how to get there.
\end{itemize}

\hypertarget{rules}{%
\paragraph{Rules}\label{rules}}

\begin{itemize}
\item
  You may not be commanding your partner.
\item
  The navigator may not touch the mouse or keyboard.
\item
  The driver may not ignore the navigator.
\item
  You must swap roles often (e.g. every 20 minutes)
\item
  Try to keep a conversation going.
\end{itemize}

\hypertarget{conversations-between-driver-and-navigator}{%
\subsubsection{Conversations between driver and
navigator}\label{conversations-between-driver-and-navigator}}

\begin{itemize}
\item
  Good pair programming is not without communication and talk.
\item
  Talk together as a pair all the time about what is happening on the
  screen.
\item
  Reflect on what you have done and where you are going.
\item
  The driver tells what they are doing, and what is happening.
\item
  The navigator comments to ensure they are doing the right thing and
  tells them what to do now and later.
\end{itemize}

\hypertarget{examples}{%
\paragraph{Examples:}\label{examples}}

\begin{itemize}
\item
  Driver: "Now we create a new function to draw a sunflower."
\item
  Driver: "Now we test if XYZ works before we continue."
\item
  Navigator: "How can you do that?"
\item
  Navigator: "Can you explain what you are doing?"
\end{itemize}

\end{document}
